\documentclass{article}

%\usepackage[T1]{fontenc}
\usepackage{inconsolata}
\usepackage{courier}
\usepackage{amsmath}
\usepackage{tcolorbox}
\usepackage[T1]{fontenc}
\usepackage[utf8]{inputenc}
\usepackage[polish]{babel}

\usepackage{color}

\definecolor{pblue}{rgb}{0.13,0.13,1}
\definecolor{pgreen}{rgb}{0,0.5,0}
\definecolor{pred}{rgb}{0.9,0,0}
\definecolor{pgrey}{rgb}{0.46,0.45,0.48}

\usepackage{listings}
\lstset{language=C++,
  showspaces=false,
  showtabs=false,
  breaklines=true,
  showstringspaces=false,
  breakatwhitespace=true,
  commentstyle=\color{pgreen},
  keywordstyle=\color{pblue},
  stringstyle=\color{pred},
  basicstyle=\color{pgreen}\ttfamily,
  %basicstyle=\ttfamily,
  moredelim=[il][\textcolor{pgrey}]{$$},
  moredelim=[is][\textcolor{pgrey}]{\%\%}{\%\%}
}


\title{Music player, dokumentacja}
\author{Katarzyna Nowicka, Łukasz Rodak, Henryk Nowakowski}
\date{ }










\begin{document}


\begin{titlepage}

{\fontfamily{roman}\selectfont

   \vspace*{\stretch{1.0}}
   \begin{center}
      \Huge\textbf{Music player}\\
      \huge\textit{Dokumentacja}\\~\\
      \large\textit{Katarzyna Nowicka, Łukasz Rodak, Henryk Nowakowski}
      
   \end{center}
   \vspace*{\stretch{2.0}}
   
   }
\end{titlepage}


\tableofcontents

\newpage
 
\section{Skąd wziął się pomysł}
 
This is the first section.
 
Lorem  ipsum  dolor  sit  amet,  consectetuer  adipiscing  
elit.   Etiam  lobortisfacilisis sem.  Nullam nec mi et 
neque pharetra sollicitudin.  Praesent imperdietmi nec ante. 
Donec ullamcorper, felis non sodales...
 
%\addcontentsline{toc}{section}{Unnumbered Section}
%\section*{Unnumbered Section}
 
%Lorem ipsum dolor sit amet, consectetuer adipiscing elit.  

 
\section{Główne założenia}
 
Lorem ipsum dolor sit amet, consectetuer adipiscing elit.  
Etiam lobortis facilisissem.  Nullam nec mi et neque pharetra 
sollicitudin.  Praesent imperdiet mi necante...

\section{Wygląd aplikacji}
 
Lorem ipsum dolor sit amet, consectetuer adipiscing elit.  
Etiam lobortis facilisissem.  Nullam nec mi et neque pharetra 
sollicitudin.  Praesent imperdiet mi necante...

\section{Instrukcja obsługi}
 
Lorem ipsum dolor sit amet, consectetuer adipiscing elit.  
Etiam lobortis facilisissem.  Nullam nec mi et neque pharetra 
sollicitudin.  Praesent imperdiet mi necante...

\section{FAQ}
 
Lorem ipsum dolor sit amet, consectetuer adipiscing elit.  
Etiam lobortis facilisissem.  Nullam nec mi et neque pharetra 
sollicitudin.  Praesent imperdiet mi necante...

\addcontentsline{toc}{section}{Unnumbered Section}
\section*{Dla deweloperów}
 
Część dla deweloperów

\section{Wstęp do QT}
 
W celu napisania dobrej, szybkiej oraz funkcjonalnej aplikacji poszukiwaliśmy odpowiedniego środowiska rozszerzającego zwykłego C++-a. Wybór padł na QT. Jest to zestaw wielu bibliotek i narzędzi programistycznych, który między innymi współpracyje z C++-em. Pozwala na budowanie aplika

\secgraficznym interfejsem użytkownika, które działają na wielu platformach, jak np. MacOS, Linux, Windows. Qt składa się z wielu przydatnych i przez nas wykorzystanych narzędzi takich jak: moc (Meta Object Compiler) czyli preprocesor generujący z plików nagłówkiwych (*.h) dodatkowe pliki źródłowe (*.cpp); uic (User Interface Compiler) jest to kompilator plików *.ui, czyli plików xml zawierających w sobie informacje o wyglądzie i ułożeniu poszczególnych przycisków, list itp. w oknie aplikacji; qmake czyli program do zarządzania procesem kompilacji, który tworzy plik Makefile, który zaś zawiera informacje o projekcie, plikach, które mają być kompilowane itd; Qt Designer to plikacja, która pozwala na sprawne edytowanie graficznego interfejsu; Qt Creator, czyli IDE, które wykorzystaliśmy do pisania aplikacji.



\setion{Prosta apcji z likacja napisana przy pomocy QT} %W sumie nie wiem czy potrzebne  
 
Lorem ipsum dolor sit amet, consectetuer adipiscing elit.  
Etiam lobortis facilisissem.  Nullam nec mi et neque pharetra 
sollicitudin.  Praesent imperdiet mi necante...

\section{Opis ważnych części kodu}
 
Przykładowo: \\

Metoda obiektu \verb|welcome_window|, która aktywuje się po kliknięciu przycisku \verb|button_login|
\\

\begin{tcolorbox}
\begin{lstlisting}

void welcome_window::on_button_login_clicked()
{
    hide();
    logwindow = new MainWindow(this);
    logwindow->show();
}

\end{lstlisting}
\end{tcolorbox}

\section{Możliwość rozbudowy}
 
Lorem ipsum dolor sit amet, consectetuer adipiscing elit.  
Etiam lobortis facilisissem.  Nullam nec mi et neque pharetra 
sollicitudin.  Praesent imperdiet mi necante...
 



\end{document}

